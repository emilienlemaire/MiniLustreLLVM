\documentclass{beamer} %

%%%BASICS
\usepackage[utf8]{inputenc}
\usepackage[T1]{fontenc}
\usepackage{lmodern}
\usepackage{amsmath, amssymb}
\usepackage{semantic}
\usepackage{appendix}
\usepackage{babel}
\usepackage{xcolor} % to access the named colour LightGray
\usepackage{minted}
% \usemintedstyle{vs}

%%%START THEME SETTINGS
\usetheme{CambridgeUS}
\usecolortheme{beaver} % seahorse, dolphin, default, beaver
\usefonttheme{professionalfonts}
\useinnertheme{rectangles}
% \useoutertheme{infolines}

\definecolor{lred}{RGB}{255,245,245}
\definecolor{mred}{RGB}{200,40,40}
\definecolor{dred}{RGB}{170,0,0}

\setbeamercolor{block title}{fg=white,bg=dred}
\setbeamercolor{block body}{bg=lred}
\setbeamercolor{section in toc}{fg=alerted text.fg}
\setbeamercolor{section in toc shaded}{bg=structure!20, fg=mred}
\setbeamercolor{section number projected}{fg=mred, bg=mred,fg=white}
\setbeamercolor{subsection number projected}{fg=mred, bg=mred,fg=white}
\setbeamertemplate{itemize item}{\color{mred}$\blacksquare$}
\setbeamertemplate{enumerate item}{\color{mred}$\blacktriangleright$}
\setbeamertemplate{enumerate item}{\color{mred}$\blacklozenge$}

\setbeamercovered{transparent}
%%%END THEME SETTINGS


%------------------------------------------------------
\title[Luste $\mapsto$ LLVM]{Compilation de Mini-Lustre vers LLVM}
\institute[UPSaclay]{Université Paris-Saclay}
\author{Lemaire \& Patault}
\date{\today}
%------------------------------------------------------

\newcommand{\ocaml}[1]{\mintinline{ocaml}{#1}}

\begin{document}
%------------------------------------------------------

\begin{frame}
    \titlepage
\end{frame}

\section{Introduction}

\subsection{TopLevel Ocaml}

\begin{frame}[fragile]{Comment vérifier le typage ?}
  \begin{minted}[baselinestretch=1.2,fontsize=\footnotesize]{llvm}
define dso_local i32 @main() #0 {
  %1 = alloca i32, align 4
  %2 = alloca i32, align 4
  %3 = alloca i32, align 4
  store i32 1, i32* %1, align 4
  store i32 1, i32* %2, align 4
  %4 = load i32, i32* %1, align 4
  %5 = load i32, i32* %2, align 4
  %6 = add nsw i32 %4, %5
  store i32 %6, i32* %3, align 4
  %7 = load i32, i32* %3, align 4
  %8 = call i32 (i8*, ...)
        @printf(i8* getelementptr inbounds
                    ([4 x i8], [4 x i8]* @.str, i64 0, i64 0), i32 %7)
  ret i32 0
}
   \end{minted}
\end{frame}

\begin{frame}
    \tableofcontents[currentsection]
\end{frame}

\end{document}
